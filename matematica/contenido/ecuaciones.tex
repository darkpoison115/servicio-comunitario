
\section{Ecuaciones}
\subsection{Ecuaciones}
Una ecuación es una igualdad entre dos expresiones que contienen una o mas
variables.  Esta es una expresión algebraica conformada una o mas variables y
operadores numéricos; estos se relacionan mediante operaciones como
suma, resta, multiplicación, división, potenciación, entre otras; y el objetivo
es conseguir el valor, o valores, de las variables que satisfacen la igualdad.
Una ecuación tiene la forma:
$$ \colorbox{RojoClaro}{13$x$ - 9}= \colorbox{AzulClaro}{5 + $x$} $$

Donde cada color representa un miembro de la ecuación, $x$ es la variable y se
puede observar que en cada miembro hay operaciones  suma, resta, y multiplicación


Para resolver una ecuación se procede a \comillas{despejar} la variable, este
proceso consiste
en agrupar todas las variables de un lado y todos los números del otro, realizar
las operaciones necesarias para simplificar y que de esta forma quede una única
variable igualada a un numero o una expresión. (por expresión se refiere a algún
tipo de función, como el valor absoluto, una raíz o una dependencia de otra variable
de forma que el resultado sea mas que un único dígito). Para resolver ecuaciones
se puede seguir la formula, algoritmo:


\begin{itemize}
    \item Colocar variables de un lado y los  números del otro. Esto se hace
        siguiendo ciertas \textbf{reglas}:
        \begin{itemize}
            \item Las \textbf{sumas} pasan como \textbf{restas}.
            \item Las \textbf{restas} pasan como \textbf{sumas}.
            \item Las \textbf{multiplicaciones} pasan como \textbf{divisiones}.
                \textbf{PERO}, tienen que estarse multiplicando a todo ese lado
                de la igualdad.
            \item Las \textbf{divisiones} pasan como \textbf{multiplicaciones}.
                \textbf{PERO}, tienen que estar dividiendo a todos los elementos
                de ese lado de la igualdad
            \item Los \textbf{exponentes} como raíces.\textbf{PERO}, tienen que
                estar abarcando a todos los elementos
                de ese lado de la igualdad
            \item Las \textbf{raíces} como exponentes. \textbf{PERO}, tienen que
                estar abarcando a todos los elementos
                de ese lado de la igualdad
        \end{itemize}
        Es decir se convierte en su \textbf{función inversa}.

    \item realizar las operaciones necesarias a ambos lados, son independientes,
        para simplificar la ecuación. \textbf{Cabe resaltar: }que las operaciones
        deben de realizarse siguiendo las reglas del álgebra, esto significa:
        \begin{enumerate}
            \item Paréntesis
            \item Exponentes
            \item Multiplicación y división
            \item Suma
        \end{enumerate}

        y si las operaciones tienen la misma jerarquía, del mismo nivel-tipo, se
        resuelven de izquierda a derecha.
        \textbf{Recordemos: } que hay un tipo especial de operación dentro de los
        exponentes, este se llama \refname{producto-notable}, y ocurre cuando la
        variable esta entre paréntesis en una operación de \textbf{suma o resta}
        y el paréntesis esta elevado a un exponente. Si no recuerda como resolverlo
        vaya a \refname{producto-notable}.
    \item En este punto deben de quedar expresiones sencillas a ambos lados de
        la igualdad, si no es que solo quedan números y una variable. por lo tanto:

        \textbf{Si la variable esta sola}: La ecuación esta resuelta(si la variable
        esta sola pero negativa se invierten los signos de la igualdad).

        \textbf{Si la variable esta siendo multiplicada o dividida}: se despeja
        para dejarla sola, se resuelve la operación resultante y este seria el
        resultado final
\end{itemize}

    Ejemplos: Resolver las siguientes ecuaciones:

    \begin{align*}
        13x - 9 &=5 + x		\\
        13x -x &= 5 +9\\
        12x &= 14 \\
        x &= \frac{14}{12} \\
        x&= \frac{7}{6}
    \end{align*}

    \begin{align*}
        124 + 18x -40 &= (19+45x)\times2 		\\
        \frac{124+18x-40}{2}&= 19 + 45x\\
        \frac{124}{2} +\frac{18x}{2} - \frac{40}{2}  &= 19 +45x\\
        \frac{18x}{2} -45x &= 19-\frac{124}{2} +\frac{40}{2} \\
        9x -45x &= 19-62+20\\
        -36x &=-23\\
        x &= \frac{-23}{-36} \\
        x &=\frac{23}{36}
    \end{align*}

    \begin{align*}
        (25x + 48)^2 -10 &= 666 \\
        (25x + 48)^2  &= 666+10 \\
        25x +48 &= \sqrt{676}\\
        25x + 48 &= 26 \\
        25x &= 26-48\\
        x &=  \frac{-22}{25}\\
    \end{align*}


\subsubsection{Forma canónica de la ecuación}
    Este es un caso particular de las ecuaciones, y es de suma importancia para
    el estudio y la resolución de las mismas. Se dice que una ecuación esta en
    su \textbf{forma canónica} cuando uno de sus miembros, lados, es 0 y el otro
    miembro no puede ser simplificado mas, es decir
    hay una expresión igualada a 0.

    La importancia de este tipo de ecuaciones es que facilitan la resolución en
    ciertos casos y en otros permiten la interpretación de un fenómeno físico con
    mayor claridad. \textbf{Recordemos} que las matemáticas son utilizadas para
    describir el comportamiento y la evolución de fenómenos físicos para de esta
    forma entender a mayor profundidad el mundo que nos rodea.

    La forma de resolver una ecuación en forma canónica es la misma que la
    explicada anteriormente. y para colocar una ecuación a su forma canónica basta
    con pasar todos los elementos a un mismo lado,siguiendo las reglas del despeje.
    Ejemplos:

    Llevando las ecuaciones anteriores a su forma canónica:

    \begin{align*}
        13x - 9 &= 5 + x		\\
        13x -x -9 -5 &= 0\\
        12x -14 &= 0 \\
    \end{align*}

    \begin{align*}
        124 + 18x -40 &= (19+45x)\times2 		\\
        \frac{124+18x-40}{2}&= 19 + 45x\\
        \frac{124}{2} +\frac{18x}{2} - \frac{40}{2}  &= 19 +45x\\
        \frac{124}{2} +\frac{18x}{2} - \frac{40}{2} -19 -45x  &=0\\
        62+9x-20-19-45x=0
        -36x+23 &= 0\\
    \end{align*}

    \begin{align*}
        (25x + 48)^2 -10 &= 666 \\
        (25x + 48)^2  &= 666+10 \\
        25x +48 &= \sqrt{676}\\
        25x + 48 &= 26 \\
        25x +48 -26 &= 0 \\
        25x+22 &= 0\\
    \end{align*}

\subsubsection{Tipos de ecuaciones}
    Existen muchos tipos de ecuaciones, estas se clasifican por el tipo de operaciones
    que son necesarias para resolverlas. La clasificación mas común, y la que se
    estudiara es la de \textbf{ecuaciones algebraicas}, esta recibe este nombre
    dado que para resolverla solo hacen falta operaciones algebraicas.

    \textbf{Cabe resaltar} que existen otro tipo de ecuaciones como las logarítmicas
    diferenciales, integrales y funcionales, estas no se estudiaran ya que
    pertenecen a cursos mas avanzados.

    Las \textbf{ecuaciones algebraicas}, se suelen sub dividir según \textbf{el
    grado} del polinomio en:

    \begin{itemize}
        \item De primer grado o lineales.
        \item De segundo grado o cuadráticas.
        \item De tercer grado o cubicas.\\
        \vdots
        \item De grado n, n $\in\mathbb{N}$.
    \end{itemize}
    \textbf{recordemos} que el grado de un polinomio es
    \textbf{el exponente mas alto al que esta elevado la variable independiente}.

    \subsubsection*{Ecuaciones 1er orden} \label{Ecuaciones-1er-orden}
    Su forma canónica es: $ax+b=0;\ a,b\in\mathbb{R}$ $a$ es llamado coeficiente
    de $x$ y $b$ es el termino independiente.

    Este tipo de ecuaciones se resuelven de la forma previamente estudiada, es
    decir se despeja la x siguiendo las reglas anteriormente nombradas. Ejemplo:

    \begin{align*}
        10x +50 &= 3		\\
        10x &= 3-50 \\
        x &= \frac{-47}{10} \\
        x &= -4.7\\
    \end{align*}

   \begin{align*}
       (95x -10)^3 &= 27		\\
       95x -10 &= \sqrt[3]{27} \\
       95x &= 3 +10\\
       x &= \frac{13}{95} \\
   \end{align*}



    \subsubsection*{Ecuaciones 2do orden} \label{Ecuaciones-2do-orden}
    Su forma canónica es: $a_1x^2+a_2x+b=0;\ a,b\in\mathbb{R}$ $a_1,a_2$ son
    llamados coeficientes  de $x$ y $b$ es el termino independiente.

    Para resolver este tipo de ecuación se busca expresarlo en su forma canónica
    y aplicar una formula, a esta se le suele llamar \textbf{resolvente cuadrática},
    o solo \textbf{resolvente} pero al ser tan utilizada se le conoce de muchas
    formas\dots

    \textbf{Cabe resaltar que:}
    también se puede escribir la forma canónica de la forma:
    $ax^2+bx+c=0$, es lo mismo, pero la formula que se usa para resolver este tipo
    de ecuaciones suele llevar $a,b,c$.

    La formula a usar es:

    $$x = \frac{-b \pm \sqrt{b^2 - 4ac}}{2a} $$


    donde $a,b,c$ son respectivamente los coeficientes de la formula.

    También se observa que existe un símbolo $\pm$ este indica que pueden haber
    \textbf{2 resultados posibles que satisfacen la ecuación}. Para conseguirlos
    se tiene que resolver 2 veces todos los cálculos, \textbf{1 vez sumando} y
    \textbf{1 vez restando}, es decir, $\pm$ se va a reemplazar por un + para
    $x_1$ y por un - para $x_2$.    Ejemplos:



    \begin{align*}
        x^2 + 5x -14 &= 0		\\
        \text{identificamos que: }& a=1,b=5,c=-14\\
        \text{entonces, procedemos a }&\text{sustituir en la formula:}\\
        x &= \frac{-5\pm \sqrt{5^2-4\times1\times-14}}{2\times1}\\
        x &= \frac{-5\pm \sqrt{25+56}}{2}\\
        x &= \frac{-5\pm \sqrt{81}}{2}\\
    \end{align*}
        Luego, realizamos las 2 operaciones:
     \begin{align*}
         x_1 &= \frac{-5 +\sqrt{81}}{2} &  x_2 &= \frac{-5 -\sqrt{81}}{2}\\
         x_1 &= \frac{-5 +9}{2} & x_2 &= \frac{-5 -9}{2}\\
         x_1 &= \frac{4}{2} &  x_2 &= \frac{-14}{2}\\
         x_1 &= 2 &  x_2 &= -7
    \end{align*}

    Algunas veces la raíz interna no es exacta, en esos casos se deja expresado:
    \begin{align*}
        x^2 + 5x -9 &= 0		\\
        \text{identificamos que: }& a=1,b=5,c=-9\\
        \text{entonces, procedemos a }&\text{sustituir en la formula:}\\
        x &= \frac{-5\pm \sqrt{5^2-4\times1\times-9}}{2\times1}\\
        x &= \frac{-5\pm \sqrt{25+36}}{2}\\
        x &= \frac{-5\pm \sqrt{61}}{2}\\
        \text{y entonces, queda }&\text{de la forma:}\\
        x_1 = \frac{-5 +\sqrt{61}}{2}\ ;&\ \ \  x_2 = \frac{-5 -\sqrt{61}}{2}
    \end{align*}

    En otras ocasiones se tendrá que despejar antes de aplicar la formula:

    \begin{align*}
        4x +186 &= -10x + 5x^2 \\
        -5x^2 +4x+10x+186 &= 0\\
        -5x^2 +14x+186 &= 0\\
        \text{identificamos que: }& a=-5,b=14,c=186\\
        \text{entonces, procedemos a }&\text{sustituir en la formula:}\\
        x &= \frac{-14\pm \sqrt{14^2-4\times-5\times186}}{2\times-5}\\
        x &= \frac{-14 \pm \sqrt{196+3720}}{-10}\\
        x &= \frac{-14 \pm \sqrt{3916}}{-10} \\
        \text{y entonces, queda }&\text{de la forma:}\\
        x_1 = \frac{-14+\sqrt{3916}}{-10}\ ;&\ \ \  x_2 = \frac{-14 -\sqrt{3916}}{-10}
    \end{align*}

    Y en algunos casos la raíz se hará 0 y se terminara con que ambos valores
    serán el mismo:

    \begin{align*}
        x^2 +4x x_1=x_2&=-2+4 &= 0		\\
        \text{identificamos que: }& a=1,b=4,c=4\\
        \text{entonces, procedemos a }&\text{sustituir en la formula:}\\
        x &= \frac{-4\pm \sqrt{4^2-4\times1\times4}}{2\times1}\\
        x &= \frac{-4 \pm \sqrt{ 16 - 16}}{2} \\
        x &= \frac{-4\ \pm 0}{2}\\
        \text{ y por lo tanto }&\text{$x_1$ y $x_2$ valen: -2}\\
        x_1=&x_2=-2
    \end{align*}




    \subsubsection{Ecuaciones de 3er orden o superior} \label{Ecuaciones-de-3er-orden-o-superior}
    Su forma canónica es: $a_1x^3+a_2x^2+a_3+b=0;\ a,b\in\mathbb{R}$ $a_1,a_2,a_3$ son
    llamados coeficientes  de $x$ y $b$ es el termino independiente.

    La forma general es:
    $$a_1x^n+a_2x^{n-1}+a_3x^{n-2}+\cdots+a_{n+1}x+b = 0$$
    donde, $a_1,a_2,a_3,\cdots,a_{n+1}$ son los coeficientes, $b$ es el termino
    independiente y todos estos son números reales. $n$ es llamado exponente máximo,
    representa el grado y puede ser cualquier numero natural ($n \in \mathbb{N}$)


    Este tipo de ecuaciones se resuelven al factorizar o utilizando el método
    de \textbf{Ruffini} ya que las raíces conseguidas son los resultados de la
    ecuación. Véase (\refname{Factorización}) para una explicación mas profunda.



\subsection{Inecuaciones}




\subsection{Sistema de Ecuaciones}
