\section{Funciones}

    Una funcion, como definicion formal, es una regla que asigna a todo
    elemento en el conjunto de partida
    un \textbf{unico} elemento en el conjunto de llegada. En si, esta puede verse
    como la relacion que hay entre dos conjuntos de numeros, y se dice que
    una magnitud, numero, es funcion de otra si el valor de la primera depende
    del valor de la segunda. De esta forma, se obtienen 3 partes fundamentales
    de las funciones:

    \begin{itemize}
        \item Conjunto de partida, dominio:

            Es el conjunto formado por todos los posibles valores que puede \textbf{tomar}
            una funcion. Corresponde a la \textbf{variable independiente}.

        \item Conjunto de llegada, rango

            Es el conjunto formado por todos los posibles valores que puede \textbf{dar
            como resultado} una funcion, estos dependen del dominio y la expresion
            que rigen la funcion. Corresponde a la \textbf{variable dependiente}.

        \item Ecuacion la cual rige la funcion (formula $f(x)= expresion$)

            Esta es una ecuacion, la cual puede ser resuelta para una serie de valores
            (dominio) y da como resultado, un conjunto de valores (rango), y cada
            valor en el dominio puede dar un unico elemento en el rango.
    \end{itemize}

    Una funcion suele tener la siguiente forma:

    $$ variable_dependiente = expresion $$

    Donde la expresion viene dada en terminos de la variable independiente (una letra), es
    decir, $ expresion= $  \textbf{ecuacion en terminos de la variable independiente}

    y la variable dependiente puede ser una letra (diferente a la variable independiente)
    o esto: $ f(varible_independiente) $ una $ f $ que representa funcion, y entre
    corchetes, la variable independientes. Algunos ejemplos son:

    \textbf{Para identificarlas:} La letra que esta sola, a un lado de la
    igualdad es la variable dependiente y la que esta siendo multiplicada, dividida,
    sumada, dentro de una raiz, etc. Es la variable independiente

    $$ f(x) = x^2 $$
    con:

    $ f(x) $: variable dependiente.

    $x$: variable independiente.

    $$ y = x+8 $$
    con:

    $ y $: variable dependiente.

    $x$: variable independiente.

    $$ f(g) = g-95 $$
    con:

    $ f(g) $: variable dependiente.

    $g$: variable independiente.

    $$ h = 975x $$
        con:

    $ h$: variable dependiente.

    $x$: variable independiente.

\subsubsection*{Operaciones con funciones} \label{Operaciones_con_funciones}

    Las funciones pueden ser sumadas, restadas, multiplicadas, divididas
    y evaluadas.


\subsubsection*{Suma y resta de funciones} \label{Suma_de_funciones}

La suma de dos funciones consiste en sumar/restar sus miembros algebraicamente y el
dominio resultante vendra dado por la interseccion de los dominios de las
funciones individuales.

Sean: $ f(x),\ g(x) $ funciones cualesquiera, con sus dominios $ Dom_f,\ Dom_g $

    $$ f(x)\pm g(x)=(f\pm g)(x)= miembros_f + miembros_g $$

y el dominio resultante va a ser dado por:


    $$ Dom_{f\pm g}= Dom_f \cap Dom_g $$


    Ejemplos:

    $ f(x)= 45x,\ Dom_f=\{4,5,8,70,100\} $

    $ g(x)=4 +95x,\ Dom_g= \{1,2,3,5,8,10,100\}$
\\

    $ f(x)+g(x) $

    \begin{align*}
        f(x)+g(x) &= 45x+4+95x 		\\
        f(x)+ g(x)&= 140x + 4
    \end{align*}

    y el dominio viene dado por:
    $$ Dom_{f+g} = Dom_f \cap Dom_g =\{4,5,8,70,100\} \cap \{1,2,3,5,8,10,100\}=\{5,8,100\} $$


    $ f(x)-g(x) $

    \begin{align*}
        f(x)-g(x) &= 45x-(4+95x) 		\\
        f(x)- g(x)&= 45x - 4 -95x\\
        f(x)- g(x)&= -50x - 4
    \end{align*}

    y el dominio viene dado por:
    $$ Dom_{f-g} = Dom_f \cap Dom_g =\{4,5,8,70,100\} \cap \{1,2,3,5,8,10,100\}=\{5,8,100\} $$


    $ g(x)-f(x) $

    \begin{align*}
        g(x)-f(x) &= 45x-(4+95x) 		\\
        g(x)-f(x)&= 45x - 4 -95x\\
        g(x)-f(x)&= -50x - 4
    \end{align*}

    y el dominio viene dado por:
    $$ Dom_{g-f} = Dom_f \cap Dom_g =\{4,5,8,70,100\} \cap \{1,2,3,5,8,10,100\}=\{5,8,100\} $$

\subsubsection*{Multiplicación y division de funciones} \label{Multiplicacion_de_funciones}

La multiplicacion/division de funciones se da de la misma forma que la suma o
resta, es decir se multiplican o dividen los terminos.

La multiplicacion seguira las propiedades asociativa y/o conmutativa para simplificarse.

En el caso de la division, esta se expresa como una fraccion, siendo numerador
y denominador dividendo y divisor respectivamente. Luego se procede a simplificar
si es posible.

Los dominios vienen dados por la interseccion en la multiplicacion y la interseccion
con la exclusion de los puntos donde el denominador se haga 0 en la division
(ya que la division por 0 no esta definida).


Ejemplos:

Sean: $ f(x),\ g(x) $ funciones cualesquiera, con sus dominios $ Dom_f,\ Dom_g $

    $$ f(x) \times g(x)=(f \times g)(x)= miembros_f \times miembros_g $$

y el dominio resultante va a ser dado por:
    $$ Dom_{f\times g}= Dom_f \cap Dom_g $$

    $$ \frac{f(x)}{g(x)}=( \frac{f}{g})(x)= \frac{miembros_f}{miembros_g} $$

y el dominio resultante va a ser dado por:
    $$ Dom_{\frac{f}{g}}= Dom_f \cap Dom_g -\{x|g(x)=0\}$$

    Ejemplos:

    $ f(x)= 45x,\ Dom_f=\{0,-4/95,4,5,8,70,100\} $

    $ g(x)=4 +95x,\ Dom_g= \{0,-4/95, 1,2,3,5,8,10,100\}$


    $ f(x)\times g(x) $

    \begin{align*}
        f(x)\times g(x) &= 45x\times(4+95x)		\\
        f(x)\times g(x) &= 45x\times4 + 45x\times95x\\
        f(x)\times g(x) &= 180x+4275x^2
    \end{align*}

    y el dominio viene dado por:

    $$ Dom_{g\times f} = Dom_f \cap Dom_g =\{0,-4/95,4,5,8,70,100\} \cap \{0,-4/95,1,2,3,5,8,10,100\}=\{0,-4/95,5,8,100\} $$

    $ \frac{f(x)}{g(x)}  $

    \begin{align*}
        \frac{f(x)}{g(x)} &= \frac{45x}{4+95x} 		\\
    \end{align*}

    y el dominio viene dado por:
    $$ Dom_{\frac{f}{g}}= Dom_f \cap Dom_g -\{x|g(x)=0\}$$
    $$ Dom_{\frac{f}{g} } =  (\{0,-4/95,4,5,8,70,100\} \cap \{0,-4/95,1,2,3,5,8,10,100\}) -\{x|4+95x=0\rigtharrow -4/95 \}$$

    $$Dom_{\frac{f}{g} }=\{0,5,8,100\} $$

$ \frac{g(x)}{f(x)}  $

    \begin{align*}
        \frac{g(x)}{f(x)} &= \frac{4+95x}{45x}		\\
        \frac{g(x)}{f(x)} &= \frac{4}{45x} +\frac{95}{45}
    \end{align*}

    y el dominio viene dado por:
    $$ Dom_{\frac{g}{f}}= Dom_f \cap Dom_g -\{x|f(x)=0\}$$
    $$ Dom_{\frac{g}{f} } =  (\{0,-4/95,4,5,8,70,100\} \cap \{0,-4/95,1,2,3,5,8,10,100\}) -\{x|45x=0\rigtharrow 0 \}$$

    $$Dom_{\frac{f}{g} }=\{-4/95,5,8,100\} $$

\subsubsection*{Evaluacion de funciones} \label{Evaluacion_de_funciones}

Es una de las operaciones mas importantes, consiste en sustituir dentro de la
expresion, la variable independiente por un elemento del dominio, esta operacion
puede repetirse tantas veces como haga falta con todos los elementos del dominio
para obtener el rango de la funcion.

Para evaluar una funcion se escribe de la siguiente forma $ f(a) $, la funcion
evaluada en le punto $a$, donde $a$ puede ser cualquier elemento o expresion
del dominio de la funcion.

\textbf{Para realizar esta operacion}, se debe sustituir la variable independiente
en la expresion de la funcion por el termino deseado y luego simplificar de ser
posible.

    Ejemplos:

    Sean:
    $f(x)= 45x$,
    $ g(x)=154 + 75x $  ,
    $ h(x)=\frac{45x}{2} - x^2 $  ,
    $ z(x) = \sqrt[3]{5x+9}  $  .

    Evaluarlas en {0,5,a-2}:

    $f(x)= 45x$

    0: $ f(0)= 45(0)=0$

    5: $ f(5)=45(5)=225 $

    a-2: $ f(a-2)= 45(a-2)=45a-90 $


    $ g(x)=154 + 75x $

    0: $ g(0)=154+75(0)=154 $

    5:  $ g(5)=154+75(5)=154+375=529 $

    a-2: $ g(a-2)=154+75(a-2)=154+75a-150=4+75a $


    $ h(x)=\frac{45x}{2} - x^2 $

    0: $ h(0)=\frac{45(0)}{2} - 0^2 = 0$


    5: $ h(5)=\frac{45(5)}{2} - 5^2 =\frac{225}{2}- 25=\frac{175}{2}  $


    a-2: $ h(a-2)=\frac{45(a-2)}{2} - (a-2)^2= \frac{45a-90}{2} -(a^2-4a+4)=\frac{-2a^2+53a-98}{2}  $


    $ z(x) = \sqrt[3]{5x+9}  $

    0:    $ z(0) = \sqrt[3]{5(0)+9}= \sqrt[3]{9}  $

    5:    $ z(5) = \sqrt[3]{5(5)+9}= \sqrt[3]{34}  $

    a-2:    $ z(a-2) = \sqrt[3]{5(a-2)+9}   = \sqrt[3]{5a-10+9} = \sqrt[3]{5a-1}$

\subsection{Función Lineal}
\subsection{Función Cuadratica}
\subsection{Función Polinomica}
\subsection{Conicas}
\subsection{Función Exponencial}
\subsection{Función Logaritmica}
\subsection{Funciones Trigonometricas}

